\documentclass{article}
\usepackage{ctex}
\usepackage{amsmath}
\usepackage{geometry}
\geometry{a4paper,left=2cm,right=2cm,top=2cm}
\title{Discrete Mathematics Quiz 2\\\small{2022 - 2023 春夏学期}}
\author{Shd0wash}
\date{\today}
\setlength{\parskip}{1em}
\setlength{\baselineskip}{20pt}
\setlength{\parindent}{0em}
\begin{document}

\maketitle

1. (31\% total)  

    (a) Find the next larger permutation in lexicographic order after 276154389. (4\%)\\  
    (b) Find the next 5-combination of the set \{1, 2, 3, 4, 5, 6, 7, 8, 9\} after \{1, 3, 5, 7, 9\}. (4\%)\\  
    (c) Find $3^{2023} \bmod 1997$. (4\%)  \\
    (d) Find the value of $\binom{-3}{4}$. (4\%)  \\
    (e) Use the construction in the proof of the Chinese remainder theorem to find all solutions to the system of congruences $ x \equiv 1 \pmod 2$, $x \equiv 2 \pmod 3$, $x \equiv 3 \pmod 5$, $x \equiv 4 \pmod 7$. (5\%)  \\
    (f) Find the coefficient of $x^{6}y^{4}$ in the expansion of $(x/3-2/x+y)^{12}$. (5\%)  \\
    (g) Write the first 6 terms of the sequence determined by the generating function: $(1-x)/(1+x)$. (5\%)  

2. (32\% total, 4\% each)  

    (a) How many ways are there to distribute 4 balls into 6 boxes, both the balls and boxes are labeled?  \\
    (b) How many ways are there to distribute 4 balls into 6 boxes, if each box must have at most one ball in it, and the balls are unlabeled but the boxes are labeled?  \\
    (c) How many ways are there to distribute 6 balls into 6 boxes, both the balls and boxes are labeled, and no box is empty?\\
    (d) How many ways are there to distribute 6 balls into 6 boxes, both the balls and boxes are labeled, and exactly 3 boxes are not empty?  \\
    (e) How many ways are there to distribute 6 balls into 6 boxes, if the balls are unlabeled, but the boxes are labeled, and exactly 3 boxes are not empty?  \\
    (f) There 4 kinds of balls colored with Red, Green, Blue and White. The number of each kind is unlimited, and the balls with the same type are unlabeled. How many ways are there to distribute 6 balls into 6 labeled boxes, if there are one or more red balls among them and no box is empty?  \\
    (g) Given 4 kinds of balls, the number of each kind is unlimited and the balls with the same type are labeled. How many ways are there to distribute 6 balls into 6 labeled boxes, if there are exactly 3 kinds of balls and no box is empty?  \\
    (h) There are 3 kinds of balls, and each kind has 3 balls. the balls with the same type are unlabeled. How many ways are there to take out 6 balls?  

3. Given 2023 cups numbered from 1 to 2023, all are placed with opening upwards. Starting from \textit{k} = 2 to 2023 each time (that is: for \textit{k} = 2; \textit{k} <= 2023; \textit{k}++), we invert all cups whose number is a multiple of \textit{k}. Question: In the end, how many cups are still opening upwards? (7\%)  

4. Find the solution to the following iteration relation: (10\%)
\[\mathit{a}_{n} = 4\mathit{a}_{n-1} - 3\mathit{a}_{n-2} + 2^{n} + 1,\mathit{a}_{0} = 1,\mathit{a}_{1} = 3\]

5. Let \textit{p} be a prime number. Prove that there are infinite terms $\mathit{a}_{k}$ in the sequence $\mathit{1, 11, 111, 1111, \ldots , 11\ldots1}.\\
$($\mathit{a}_{k}$ \textit{has} \textit{k} \textit{`1's}) such that $\mathit{p} \mid \mathit{a}_{k}$. (\textit{Hint}: \textit{Fermat's little theorem}) (10\%) 

6. Given any positive integer \textit{m}, prove that a multiple of \textit{m} can be found in the Fibonacci sequence. (10\%)
\end{document}