\documentclass{ctexart}
\usepackage{amsmath, amssymb, geometry,enumitem, physics, mismath}
\geometry{a4paper,scale=0.66,top=1in,bottom=1in,left=1in,right=1in}

\title{\vspace{-4em}\textbf{数学分析(甲)II(H)2023-2024 春夏期末答案}}
\author{图灵回忆卷\quad\quad by jayi0908}

\linespread{1.6}
\addtolength{\parskip}{.2em}

\begin{document}

\maketitle

\begin{enumerate}
    \item[\textbf{一、}] \textbf{(10 分)} 定义:设 $f(x,y)$ 在 $P_0(x_0,y_0)$ 的邻域 $\mathrm{U}(P_0)$ 上有定义,对 $P(x_0+\Delta x, y_0+\Delta y)\in\mathrm{U}(P_0)$,
    若 $\Delta z = f(x_0+\Delta x, y_0+\Delta y)-f(x_0, y_0)$ 可表示为 
    $A\Delta x+B\Delta y+o(\rho)$,\\
    其中 $\rho=\sqrt{(\Delta x)^2+(\Delta y)^2}$,$A,B$ 为仅与 $P_0$ 有关的常数,则称 $f(x,y)$ 在 $P_0(x_0,y_0)$ 可微。 \\
    证明:$f(x,0)=0$,$f_x'(x,0)=0$,$f(0,y)=y\arctan{\dfrac{1}{\left\vert y \right\vert}}$,$f_y'(0,0)=\lim_{y \to 0}\arctan{\dfrac{1}{\left\vert y \right\vert }}=\dfrac{\pi}{2}$ \\
    $\Delta z=f(\Delta x,\Delta y)=\Delta y\arctan{\dfrac{1}{\sqrt{(\Delta x)^2+(\Delta y)^2}}}=\Delta y\arctan{\dfrac{1}{\rho}}.$
    \begin{align*} 
    \text{故}& \quad\lim_{\rho \to 0}\frac{\Delta z-f_x'(0,0)\Delta x-f_y'(0,0)\Delta y}{\rho} \\
    & =\lim_{\rho \to 0}\frac{\Delta y(\arctan{\frac{1}{\rho}-\frac{\pi}{2}})}{\rho} \\
    & =\lim_{\rho \to 0}\frac{\rho\sin\theta}{\rho}\cdot\lim_{\rho \to 0}(\arctan{\frac{1}{\rho}}-\frac{\pi}{2})=0
    \end{align*}
    从而 $\Delta z=f_x'(0,0)\Delta x+f_y'(0,0)\Delta y+o(\rho)$,故 $f(x,y)$ 在 $(0,0)$ 可微。
    \item[\textbf{二、}] \textbf{(32 分)}
    \begin{enumerate}
        \item[\textbf{1.}] 利用球坐标变换,设 $x=r\sin\theta\cos\varphi$,$y=r\sin\theta\sin\varphi$,$z=r\cos\theta$,
        $0\le r\le 1$,$0\le\theta\le\dfrac{\pi}{2}$,$0\le\varphi\le\dfrac{\pi}{2}$,则
        $J(r,\theta,\varphi)=\begin{vmatrix}
        x_r & x_{\theta} & x_{\varphi} \\
        y_r & y_{\theta} & y_{\varphi} \\
        z_r & z_{\theta} & z_{\varphi}
        \end{vmatrix}=r^2\sin\theta$
        \begin{align*}
        \text{故} \quad\iiint_{V}\sqrt{x^2+y^2+z^2}\dd{x}\dd{y}\dd{z} 
        & =\int_{0}^{\frac{\pi}{2}}\int_{0}^{\frac{\pi}{2}}\int_0^1r\cdot r^2\sin\theta\dd{r}\dd{\theta}\dd{\varphi} \\
        & =\int_0^{\frac{\pi}{2}}\int_0^{\frac{\pi}{2}}\frac{1}{4}\sin\theta\dd{\theta}\dd{\varphi} \\
        & =\int_0^{\frac{\pi}{2}}\frac{1}{4}\dd{\varphi}=\frac{\pi}{8}.
        \end{align*}
        \item[\textbf{2.}] $f(x)=\ds\int_0^x\sqrt{\sin t}\dd{t}$,$f'(x)=\sqrt{\sin x}$,
        \begin{align*}
        \text{故} \quad\int_L x\dd{s}
        & =\int_0^{\pi}x\cdot\sqrt{1+\sin x}\dd{x} \\
        & =\int_0^{\pi}x\sqrt{(\sin\frac{x}{2}+\cos\frac{x}{2})^2}\dd{x} \\
        & =4\int_0^{\frac{\pi}{2}}x(\sin x+\cos x)\dd{x} \\
        & =4(x\sin x-x\cos x+\sin x+\cos x)\vert_0^{\frac{\pi}{2}}=2\pi.
        \end{align*}
        \item[\textbf{3.}] $P(x,y)=e^x\sin y-y^2$,$Q(x,y)=e^x\cos y$,设 $A(0,0)$,$B(\pi,0)$,$L':L+\overline{BA}$ 为闭合曲线,
        围成闭区域 $D$,$\dfrac{\partial Q}{\partial x}=e^x\cos y$,$\dfrac{\partial P}{\partial y}=e^x\cos y-2y$. \\
        由格林公式 $$\oint_{L'}P\dd{x}+Q\dd{y}=-\iint_D 2y\dd{\sigma}=-\int_0^\pi\dd{x}\int_0^{\sin x}2y\dd{y}=-\int_0^\pi \sin^2x\dd{x}=-\frac{\pi}{2}$$ \\
        在 $AB$ 上 $P(x,y)=0$,$\dd{y}=0$,故 $\ds\oint_{AB}P\dd{x}+Q\dd{y}=0$,故 $\ds\oint_{L}P\dd{x}+Q\dd{y}=-\frac{\pi}{2}$.
        \item[\textbf{4.}] 设 $x=r\cos\theta$,$y=r\sin\theta$,$z=r$,$0\le r\le 1$,$0\le\theta\le 2\pi$. \\
        则 $\dfrac{\partial(z,x)}{\partial(r,\theta)}=\begin{vmatrix}
        1 & 0 \\
        \cos\theta & -r\sin\theta
        \end{vmatrix}=-r\sin\theta$,$\dfrac{\partial(x,y)}{\partial(r,\theta)}=\begin{vmatrix}
        \cos\theta & -r\sin\theta \\
        \sin\theta & -r\cos\theta
        \end{vmatrix}=r$,从而为负向。
        \begin{align*}
        \text{故} \quad\iint_Sy^2\dd{z}\dd{x}+(z+1)\dd{x}\dd{y}
        &=-\int_0^{2\pi}\int_0^1(r^2\sin^2\theta\cdot (-r\sin\theta)+(r+1)r)\dd{r}\dd{\theta} \\
        &=\frac{1}{4}\int_0^{2\pi}\sin^3\theta\dd{\theta}-\frac{5}{6}\int_0^{2\pi}\dd{\theta} \\
        &=\frac{1}{8}\int_0^{2\pi}\sin\theta(1-\cos 2\theta)\dd{\theta}-\frac{5\pi}{3}=-\frac{5\pi}{3}.
        \end{align*}
    \end{enumerate}
    \item[\textbf{三、}] \textbf{(10 分)} $f(x-z,y-z)=0$,故 $f_x=f_1$,$f_y=f_2$,$f_z=-(f_1+f_2)\neq 0$ \\
    由隐函数定理 $\dfrac{\partial z}{\partial x}=-\dfrac{f_x}{f_z}=\dfrac{f_1}{f_1+f_2}$,$\dfrac{\partial z}{\partial y}=-\dfrac{f_y}{f_z}=\dfrac{f_2}{f_1+f_2}$,
    故 $\dfrac{\partial z}{\partial x}+\dfrac{\partial z}{\partial y}=1$.
    \item[\textbf{四、}] \textbf{(10 分)} 即求平面上一点 $(x,y,z)$,使得 $f(x,y,z)=\sqrt{(x-x_0)^2+(y-y_0)^2+(z-z_0)^2}$ 取到最小值。 \\
    用拉格朗日乘数法,设 $F(x,y,z,\lambda)=\sqrt{(x-x_0)^2+(y-y_0)^2+(z-z_0)^2}+\lambda(ax+by+cz+d)$,有 
    $F_x=\dfrac{x-x_0}{f(x,y,z)}+\lambda a$,$F_y=\dfrac{y-y_0}{f(x,y,z)}+\lambda b$,$F_z=\dfrac{z-z_0}{f(x,y,z)}+\lambda c$,$F_\lambda=ax+by+cz+d$. \\
    令 $F_x=0$,$F_y=0$,$F_z=0$,$F_\lambda=0$,\\
    若 $abc\neq 0$,则有 $\dfrac{x-x_0}{a}=\dfrac{y-y_0}{b}=\dfrac{z-z_0}{c}:=t$,即 $x=x_0+at$,$y=y_0+bt$,$z=z_0+ct$;\\
    若 $abc=0$,比如 $a=0$,则 $x=x_0=x_0+at$,对于 $b,c$ 同理,即仍可写成 $x=x_0+at$,$y=y_0+bt$,$z=z_0+ct$. \\
    故 $x=x_0+at$,$y=y_0+bt$,$z=z_0+ct$.\\
    代入 $F_\lambda=0$ 得 $a(x_0+at)+b(y_0+bt)+c(z_0+ct)+d=0$,即 $t=-\dfrac{ax_0+by_0+cz_0+d}{a^2+b^2+c^2}$。\\
    故 $f(x,y,z)=\sqrt{(at)^2+(bt)^2+(ct)^2}=\dfrac{|ax_0+by_0+cz_0+d|}{\sqrt{a^2+b^2+c^2}}$ 为所求距离。
    \item[\textbf{五、}] \textbf{(10 分)} Dirichlet 判别法:
    \begin{enumerate}
        \item[(1)] $\{\displaystyle\sum_{k=1}^na_k(x)\}_{n=1}^{\infty}$ 在 $I$ 上一致有界;
        \item[(2)] $\forall x\in I$,$\{b_n(x)\}_{n=1}^{\infty}$ 单调;
        \item[(3)] $\{b_n(x)\}_{n=1}^{\infty}$ 在 $I$ 上一致收敛于 $0$.  
    \end{enumerate}
    满足这三点则有 $\sum a_n(x)b_n(x)$ 在 $I$ 上一致收敛。\\
    证明:$\displaystyle\sum_{k=1}^n\cos kx=\dfrac{1}{2\sin{\frac{x}{2}}}(\sin{(n+\dfrac{1}{2})x-\sin{\dfrac{x}{2}}})\le\dfrac{1}{2\sin\frac{x}{2}}-1.$ 
    故 $\sum\cos kx$ 在 $(0,2\pi)$ 上内闭一致有界。 \\
    $\{\dfrac{n}{n^2+1}\}=\{\dfrac{1}{n+\frac{1}{n}}\}$ 单调趋于 $0$,且对于以 $x$ 为变元的函数列来说相当于常函数列,故一致收敛于 $0$.\\
    故由 Dirichlet 判别法,$\displaystyle\sum_{k=1}^n\dfrac{n\cos kx}{n^2+1}$ 在 $(0,2\pi)$ 上内闭一致收敛。\\
    \item[\textbf{六、}] \textbf{(10 分)} $T=2l=2\Rightarrow l=1$,故 $$a_0=\dfrac{1}{l}\int_{-l}^{l}f(x)\dd{x}=\int_0^1x^2\dd{x}=\dfrac{1}{3},$$
    \begin{align*}
        a_n &=\dfrac{1}{l}\int_{-l}^{l}f(x)\cos\dfrac{n\pi x}{l}\dd{x} \\
        &=\int_0^1x^2\cos{n\pi x}\dd{x} \\
        &=(\dfrac{1}{n\pi}x^2\sin{n\pi x}+\dfrac{2}{n^2\pi^2}x\cos{n\pi x}-\dfrac{2}{n^3\pi^3}\sin{n\pi x})\vert_0^1 \\
        &=\dfrac{2}{n^2\pi^2}(-1)^n.
    \end{align*}
    \begin{align*}
        b_n &=\dfrac{1}{l}\int_{-l}^{l}f(x)\sin\dfrac{n\pi x}{l}\dd{x} \\
        &=\int_0^1x^2\sin{n\pi x}\dd{x} \\
        &=(\dfrac{-1}{n\pi}x^2\cos{n\pi x}+\dfrac{2}{n^2\pi^2}x\sin{n\pi x}+\dfrac{2}{n^3\pi^3}\cos{n\pi x})\vert_0^1 \\
        &=\dfrac{(-1)^{n-1}}{n\pi}+\dfrac{2}{n^3\pi^3}((-1)^n-1).
    \end{align*}
    故 $f(x)=\dfrac{1}{6}+\displaystyle\sum_{n=1}^{\infty}(\dfrac{2}{n^2\pi^2}(-1)^n\cos{n\pi x}+(\dfrac{(-1)^{n-1}}{n\pi}+\dfrac{2}{n^3\pi^3}((-1)^n-1))\sin{n\pi x})$,\\
    在 $x=\pm 1$ 时收敛于 $\dfrac{0+1}{2}=\dfrac{1}{2}$.
    \item[\textbf{七、}] \textbf{(10 分)} Cauchy 收敛准则:对于常数项级数 $\sum x_k$,其收敛的充要条件为:
    对 $\forall \varepsilon > 0$,存在 $N$,使得对 $\forall n > N$,$\forall p \in \mathbb{N}_+$,有 $|x_{n+1} + x_{n+2} + \cdots + x_{n+p}| < \varepsilon$。\\
    证明:由条件,$\forall \varepsilon_1 > 0, \exists N_1, \forall n > N_1, \forall p \in \mathbb{N}_+$,有 $|a_{n+1} + a_{n+2} + \cdots + a_{n+p}| < \varepsilon_1$;\\
    $\forall \varepsilon_2 > 0, \exists N_2, \forall n > N_2, \forall p \in \mathbb{N}_+$,有 $|b_{n+p+1}-b_{n+p}|+|b_{n+p}-b_{n+p-1}|+\cdots+|b_{n+2}-b_{n+1}| < \varepsilon_2$。\\
    则不难知存在 $M>0$ 与 $N_3$,使得对 $\forall n > N_3$,有 $|b_{n}| < M$。\\
    对 $\forall\varepsilon > 0$,取 $\varepsilon_1$、$\varepsilon_2$ 使得 $\varepsilon_1(M+\varepsilon_2)=\varepsilon$,
    取 $N = \max\{N_1, N_2, N_3\}$,则对 $\forall n > N$,$\forall p \in \mathbb{N}_+$,有:
    \begin{align*}
    &\quad|a_{n+1}b_{n+1} + a_{n+2}b_{n+2} + \cdots + a_{n+p}b_{n+p}| \\
    &\leq |b_{n+1}(a_{n+1} + a_{n+2} + \cdots + a_{n+p})| + |(a_{n+2}+a_{n+3}+\cdots+a_{n+p})(b_{n+2} - b_{n+1})| \\
        &\quad\quad + |(a_{n+3}+\cdots+a_{n+p})(b_{n+3} - b_{n+2})| + \cdots + |a_{n+p}(b_{n+p} - b_{n+p-1})| \\
    &< \varepsilon_1(|b_{n+1}|+|b_{n+2}-b_{n+1}|+\cdots+|b_{n+p}-b_{n+p-1}|) \\
    &< \varepsilon_1(M+\varepsilon_2) = \varepsilon.
    \end{align*}
    由 Cauchy 收敛准则,$\sum a_n b_n$ 收敛。\\
    注,本题也有利用 Abel 变换的做法: \\
    令 $A_n=\ds\sum_{k=1}^na_k$,$A_0=0$,则 $\ds\sum_{k=1}^na_nb_n=\ds\sum_{k=1}^n(A_k-A_{k-1})b_k=A_nb_n+\ds\sum_{k=1}^{n-1}A_k(b_k-b_{k+1})$. \\
    由于 $\sum a_n$ 收敛,故 $\{A_n\}$ 为收敛数列; \\
    由于 $\sum (b_{n+1}-b_n)$ 绝对收敛,故其部分和 $\{b_n\}$ 也是收敛数列,故 $\{A_n b_n\}$ 也是收敛数列。\\
    故即证 $\sum A_n(b_n-b_{n+1})$ 收敛。 \\
    由 Cauchy 收敛准则,对 $\forall \varepsilon_1>0,\exists N_1\in\mathbb{N}_+,\forall n>N_1,\forall p\in\mathbb{N}_+$,有 $\ds\sum_{i=1}^p|b_{n+i+1}-b_{n+i}|<\varepsilon_1$ \\
    设 $\ds\lim_{n\to\infty}A_n=A$,则 $\exists N_2\in\mathbb{N}_+,\forall n>N_2, |A_n|<\dfrac{3}{2}|A|$, \\
    对 $\forall \varepsilon>0$,取 $\varepsilon_1=\dfrac{2\varepsilon}{3|A|}$,$N=\max{N_1,N_2}$,则对 $\forall n>N, \forall p\in\mathbb{N}_+$,有 \\
    \begin{align*}
    |\ds\sum_{i=1}^pA_{n+i}(b_{n+i}-b_{n+1+i})|
    &\le\ds\sum_{i=1}^p|A_{n+i}||b_{n+1+i}-b_{n+i}| \\
    &<\dfrac{3|A|}{2}\ds\sum_{i=1}^p|b_{n+1+i}-b_{n+i}| \\
    &<\dfrac{3|A|}{2}\cdot\varepsilon_1<\varepsilon
    \end{align*}
    由 Cauchy 收敛准则,$\sum A_n(b_n-b_{n+1})$ 收敛,得证。
    \item[\textbf{八、}] \textbf{(8 分)} $$ \dfrac{\dd{g_{\theta}}}{\dd{t}}=f_1\cos\theta+f_2\sin\theta=0, $$
    \begin{align*} 
    \dfrac{\dd{^2g_{\theta}}}{\dd{t^2}}&=f_{11}\cos^2\theta+2f_{12}\cos\theta\sin\theta+f_{22}\sin^2\theta \\
    &=\dfrac{f_{11}+f_{22}}{2}+\dfrac{f_{11}-f_{22}}{2}\cos 2\theta+f_{12}\sin 2\theta > 0.
    \end{align*}
    上式对 $\forall\theta\in [0,2\pi)$ 成立,故 $f_1=f_2=0$,且由辅助角公式,$\dfrac{f_{11}+f_{22}}{2}>\sqrt{(\dfrac{f_{11}-f_{22}}{2})^2+f_{12}^2}$ \\
    即 $f_{11}\cdot f_{22}-f_{12}^2=\begin{vmatrix} f_{11} & f_{12} \\ f_{21} & f_{22}\end{vmatrix}>0$,
    且代入 $\theta=0$ 到 $\dfrac{\dd{^2g_{\theta}}}{\dd{t^2}}$ 中,得 $f_{11}>0$, \\
    利用 Hesse 矩阵的正定性,得 $f(x,y)$ 在 $(0,0)$ 处有极小值。
\end{enumerate}

\end{document}