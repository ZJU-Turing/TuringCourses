\documentclass{ctexart}
\usepackage{amsmath, amssymb, geometry,enumitem}

\geometry{a4paper,scale=0.66,top=1in,bottom=1in,left=1in,right=1in}

\title{\vspace{-4em}\textbf{线性代数 II(H)2024-2025 春夏期中答案}}
\author{图灵回忆卷\quad\quad by jayi0908}
\date{2025 年 4 月 23 日}

\linespread{1.6}
\addtolength{\parskip}{.2em}

\begin{document}

\maketitle

\begin{enumerate}
    \item[\textbf{一、}] \textbf{(10 分)} 设所求直线为 \(x=1+at, y=bt, z=2+ct\)(\(t\in\mathbb{R}\))。  \\
    由于平面 \(3x - y + 2z + 2 = 0\) 的法向量为\(\overrightarrow{n}=(3,-1,2)\),故 $3a - b + 2c = 0 \quad (1)$. \\
    已知直线参数方程:\(x=1+4s, y=-1-2s, z=s\)(\(s\in\mathbb{R}\)),  
    相交即坐标相等,解得:  
    \[at=4s,\ bt=-1-2s,\ 2+ct=s\]  
    消去\(s,t\)并结合(1),得方向向量\(\boldsymbol{v}=(12,-22,-29)\)。故所求直线为:
    \[\dfrac{x-1}{12}=\dfrac{y}{-22}=\dfrac{z-2}{-29}.\]

    \item[\textbf{二、}] \textbf{(10 分)} 证明:
    先证明 $\operatorname{range}(T-I) + \operatorname{null}(T-I) = V$:\\
    对任意\(v\in V\),由 \(T^2 - 3T + 2I = 0\) 得 \((T - I)(T - 2I) = 0\). \\
    令 \(v_1=(2I-T)v\),则 \((T-I)v_1=0\),即 \(v_1\in\text{null}(T - I)\);
    \(v_2=(T-I)v\),则 \(v_2\in\text{range}(T - I)\); \\
    显然 \(v = v_1 + v_2\),故 \(V = \text{null}(T - I) + \text{range}(T - I)\)。 \\
    再证明交集为零元:\\
    设 \(u\in\text{null}(T - I) \cap \text{range}(T - I)\),则 \(Tu = u\) 且存在 \(w\) 使 \(u=(T - I)w\),代入得 \(u=(T-I)w=Tw-w=u-w \implies w=0 \implies u=0\). \\
    综上:$V = \text{null}(T - I) \oplus \text{range}(T - I)$.
    
    \item[\textbf{三、}] \textbf{(10 分)} 对任意 \(v\in V\),设 \(v = \displaystyle\sum_{i=1}^n c_i v_i\)(\(c_i\in\mathbb{F}\)),则:  
    \[
    \psi(v) = \psi\left(\sum_{i=1}^n c_i v_i\right) = \sum_{i=1}^n c_i \psi(v_i)
    \]  
    又对偶基 \(\{\varphi_i\}\) 满足 \(\varphi_i(v_j)=\delta_{ij}\),故:  
    \[
    \left(\sum_{i=1}^n \psi(v_i)\varphi_i\right)(v) = \sum_{i=1}^n \psi(v_i)\varphi_i\left(\sum_{j=1}^n c_j v_j\right) = \sum_{i=1}^n \psi(v_i)c_i
    \]  
    两边相等,故 \(\psi = \displaystyle\sum_{i=1}^n \psi(v_i)\varphi_i\).
    \item[\textbf{四、}] \textbf{(10 分)}
    \begin{enumerate}
        \item[\textbf{1.}] 设 \(\deg p = m\),对任意\(f\in\mathcal{P}(\mathbb{C})\),由带余除法得 \(f = pq + r\)(\(r=0\) 或 \(\deg r < m\)),故 \(f + U = r + U\)。若 \(r_1 + U = r_2 + U\),则 \(r_1 - r_2 = pq\),由次数关系得 \(r_1 = r_2\),故 \(\dim(\mathcal{P}(\mathbb{C})/U) = m\)($\mathcal{P}(\mathbb{C})/U$ 的维数等价于余数多项式 $r$ 构成的空间维数,自然为 $m$).
        \item[\textbf{2.}] 次数小于 \(m\) 的多项式 \(1,x,\cdots,x^{m-1}\) 的等价类线性无关且张成商空间,故基为:\[\{1 + U,\ x + U,\ \cdots,\ x^{m-1} + U\}.\]
    \end{enumerate}
    \item[\textbf{五、}] \textbf{(10 分)} \textbf{必要性:}设 \(S,T\) 可同时对角化,则存在 \(V\) 的一组基 \(\{v_1, v_2, \ldots, v_n\}\)(设 $P=(v_1,\cdots,v_n)$),使得 \([S] = P\operatorname{diag}(\lambda_1,\cdots,\lambda_n)P^{-1}, \quad [T] = P\operatorname{diag}(\mu_1,\cdots,\mu_n)P^{-1}\),则
    \[ [ST] = [S][T] = P\operatorname{diag}(\lambda_1\mu_1,\cdots,\lambda_n\mu_n)P^{-1} = [T][S] = [TS] \]  
    因此必要性成立。

    \textbf{充分性:} 因 \(S\) 可对角化,故 \(V\) 可分解为 \(S\) 的特征子空间的直和:\(V = \displaystyle\bigoplus_{\lambda \in \sigma(S)} E(\lambda, S)\),其中 \(\sigma(S)\) 是 \(S\) 的特征值集,\(E(\lambda, S) = \{v \in V \mid Sv = \lambda v\}\) 为对应特征子空间。 \\
    任取 \(\lambda \in \sigma(S)\) 及 \(v \in E(\lambda, S)\),则 \(Sv = \lambda v\). 由 \(ST = TS\),得:  
    \[
    S(Tv) = T(Sv) = T(\lambda v) = \lambda (Tv)
    \]  
    故 \(Tv \in E(\lambda, S)\),即 \(T(E(\lambda, S)) \subseteq E(\lambda, S)\),因此 \(E(\lambda, S)\) 是 \(T\) 的不变子空间。 \\
    因 \(T\) 可对角化,其在不变子空间 \(E(\lambda, S)\) 上的限制 \(T|_{E(\lambda, S)}\) 仍可对角化。故 \(E(\lambda, S)\) 存在一组由 \(T|_{E(\lambda, S)}\) 的特征向量组成的基,记为 \(B_\lambda\)。则 \(B_\lambda \subseteq E(\lambda, S)\),\(B_\lambda\) 中的向量也是 \(S\) 的特征向量(对应特征值 \(\lambda\))。 \\
    从而令 \(B = \displaystyle\bigcup_{\lambda \in \sigma(S)} B_\lambda\),则 \(B\) 是 \(V\) 的一组基(由直和性质),且每个向量均为 \(S\) 和 \(T\) 的共同特征向量。因此 \(S,T\) 在基 \(B\) 下的矩阵均为对角矩阵,即 \(S,T\) 可同时对角化。充分性成立。得证。
    \item[\textbf{六、}] \textbf{(10 分)} 特征子空间 \(E(\lambda, A) = \ker(\lambda I - A)\),其维数为 \(n - r(\lambda I - A)\).\\
    因矩阵与其转置秩相等,即 \(r(\lambda I - A) = r(\lambda I - A^T)\),故:  
    \[
    \dim E(\lambda, A) = n - r(\lambda I - A) = n - r(\lambda I - A^T) = \dim E(\lambda, A^T).
    \]
    \item[\textbf{七、}] \textbf{(10 分)} 
    \begin{enumerate}
        \item[\textbf{(1)}] 先证明 \((W_1 + W_2)^\perp \subseteq W_1^\perp \cap W_2^\perp\):  \\
        任取 \(v\in(W_1 + W_2)^\perp\),则对 $\forall v_1\in W_1$,\(v\perp v_1\);对 $\forall v_2\in W_2$,\(v\perp v_2\),故 \(v\in W_1^\perp \cap W_2^\perp\).  
        再证明 \((W_1 + W_2)^\perp \supseteq W_1^\perp \cap W_2^\perp\):\\
        任取 \(v\in W_1^\perp \cap W_2^\perp\),对 \(w_1 + w_2\in W_1 + W_2\),\(\langle v, w_1 + w_2\rangle = \langle v, w_1\rangle + \langle v, w_2\rangle = 0\),故\(v\in(W_1 + W_2)^\perp\). 
        \item[\textbf{(2)}] 对 \(W_1^\perp, W_2^\perp\) 用 (1) 得 \((W_1^\perp + W_2^\perp)^\perp = W_1 \cap W_2\),两边取正交补即证.  
    \end{enumerate}
    \item[\textbf{八、}] \textbf{(10 分)} 
    \begin{enumerate}
        \item[\textbf{1.}] 已知 $v_1,\cdots,v_k$ 是 $C=XX^T$ 的单位特征向量,且 $C$ 是实对称矩阵。实对称矩阵的不同特征值对应的特征向量必正交(以本题为例,设 $Cv=\lambda v$,$Cw=\mu w$,则 $\lambda v^{T}w=(Cv)^{T}w=v^{T}Cw=\mu v^{T}w$,由于 $\lambda\neq\mu$ 可得 $v^{T}w=\langle v,w \rangle=0$),故:  
        \[
        \langle v_i, v_j \rangle = v_i^T v_j = \delta_{ij}
        \]  
        故 $\{v_1,\cdots,v_k\}$ 是 $V$ 的规范正交基。
        
        正交投影需满足两个条件:\emph{幂等性}($P_V^2=P_V$)和\emph{对称性}($P_V^T=P_V$)。 

        幂等性:因 $U=(v_1,\cdots,v_k)$ 的列是规范正交基,故 $U^T U=I$,因此:  
        \[
        P_V^2 = (UU^T)(UU^T) = U(U^T U)U^T = UI_k U^T = UU^T = P_V
        \]  

        对称性:  
        \[
        P_V^T = (UU^T)^T = (U^T)^T U^T = UU^T = P_V
        \]  
        且 $P_V$ 的值域为 $\operatorname{span}\{v_1,\cdots,v_k\}=V$,故 $P_V$ 是 $\mathbb{R}^n$ 到 $V$ 的正交投影。
        \item[\textbf{2.}] 对任意向量$x$和正交投影$P$,有正交分解$x=Px+(x-Px)$,故:  
        \[
        \|x\|^2 = \|Px\|^2 + \|x-Px\|^2 \implies \|x-Px\|^2 = \|x\|^2 - \|Px\|^2
        \]  
        因此总投影误差可写为:  
        \[
        \sum_{i=1}^p \|x_i - Px_i\|^2 = \sum_{i=1}^p \|x_i\|^2 - \sum_{i=1}^p \|Px_i\|^2
        \]  
        因 $\sum\|x_i\|^2$ 是常数(与 $P$ 无关),故即证:$\displaystyle\sum_{i=1}^p \|P_V x_i\|^2 \ge \displaystyle\sum_{i=1}^p \|P_W x_i\|^2.$  
        \[
        \|P_V x_i\|^2 = (P_V x_i)^T (P_V x_i) = x_i^T P_V^T P_V x_i = x_i^T P_V x_i \quad (\text{因} \, P_V^T=P_V=P_V^2)
        \]  
        故总和为:  
        \[
        \sum_{i=1}^p x_i^T P_V x_i = \operatorname{tr}\left(P_V \sum_{i=1}^p x_i x_i^T\right) = \operatorname{tr}(P_V C)
        \]  
        (注:$x^T A x = \operatorname{tr}(A x x^T) \quad (1)$,且$C=XX^T=\sum x_i x_i^T$) \\
        设 $C$ 的特征值为 $\lambda_1 \ge \lambda_2 \ge \cdots \ge \lambda_n > 0$,对应的规范正交特征向量为$v_1,\cdots,v_n$,则 $C=\displaystyle\sum_{i=1}^n \lambda_i v_i v_i^T$。
        对 $P_V=UU^T$($U=(v_1,\cdots,v_k)$),有:  
        \[
        \operatorname{tr}(P_V C) = \operatorname{tr}\left(\sum_{i=1}^k v_i v_i^T \cdot \sum_{j=1}^n \lambda_j v_j v_j^T\right) = \sum_{i=1}^k \lambda_i \quad (\text{仅当}\space i=j\space\text{时迹非零})
        \]  
        对任意 $k$ 维子空间 $W$ 的正交投影 $P_W=QQ^T$($Q$ 的列是 $W$ 的规范正交基),利用 (1) 有:$\operatorname{tr}(P_W v_i v_i^{T})=v_i^{T}QQ^{T}v_i=\|v_i^T Q\|^2$,从而:  
        \[
        \operatorname{tr}(P_W C) = \sum_{i=1}^n \lambda_i \cdot \|v_i^T Q\|^2
        \]  
        因 $\displaystyle\sum_{i=1}^n \|v_i^T Q\|^2 = \operatorname{tr}(Q^T (\displaystyle\sum_{i=1}^n v_i v_i^T) Q) = \operatorname{tr}(Q^T I Q) = k$,由特征值的排序性,最大的 $\operatorname{tr}(P_W C)$ 必为 $\displaystyle\sum_{i=1}^k \lambda_i$(当 $Q$ 的列取前 $k$ 个特征向量时达到)。

        因此 $\sum\|P_V x_i\|^2 \ge \sum\|P_W x_i\|$,得证。
    \end{enumerate}
    \item[\textbf{九、}] \textbf{(20 分)} \begin{enumerate}
        \item[\textbf{1.}] \textbf{假}:反例 \(\boldsymbol{a}=\boldsymbol{i},\boldsymbol{b}=\boldsymbol{i},\boldsymbol{c}=\boldsymbol{j}\),\((\boldsymbol{a}\times\boldsymbol{b})\times\boldsymbol{c}=\boldsymbol{0}\times\boldsymbol{j}=\boldsymbol{0}\),\(\boldsymbol{a}\times(\boldsymbol{b}\times\boldsymbol{c})=\boldsymbol{i}\times\boldsymbol{k}\neq\boldsymbol{0}\),不等。
        \item[\textbf{2.}] \textbf{真}:代数基本定理的结果。
        \item[\textbf{3.}] \textbf{假}:反例 \(T=\begin{pmatrix}0&1\\0&0\end{pmatrix}\) 不可对角化,无对角矩阵表示。
        \item[\textbf{4.}] \textbf{假}:内积需满足正定性,取 \(f(t)\) 在 \([0,\frac{1}{2}]\) 为0,在 \((\frac{1}{2},1]\) 非零,则 \(\langle f,f\rangle=0\) 但 \(f\neq0\),不是内积。  
    \end{enumerate}
\end{enumerate}

\end{document}