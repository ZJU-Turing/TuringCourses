\documentclass[UTF8,14pt,normal]{ctexart}
\usepackage{amsmath}
\usepackage{amssymb}
\usepackage{geometry}
\geometry{a4paper,scale=0.66,top=0.8in,bottom=1.5in,left=1in,right=1in}

\title{\textbf{线性代数 II(H)2024-2025 春夏期末}}
\author{图灵回忆卷\quad\quad by jayi0908}
\date{2025 年 6 月 18 日}

\linespread{1.2}
\addtolength{\parskip}{.8em}

\begin{document}

\maketitle

\noindent{\heiti\textbf{一、(16 分)}} $P_0(1,0,1)$ 到平面 $x+y+z=D\space(D>0)$ 的距离为 $\sqrt{3}$,求 $P_0$ 到直线 $\begin{cases}x+y+z=D \\ x-y+z=1\end{cases}$ 的距离.

\noindent{\heiti\textbf{二、(16 分)}} $V$ 为次数不超过 $2$ 的实数多项式空间,定义在 $V$ 上的内积为 $\langle p,q \rangle = \int_0^1 p(x)q(x)\mathrm{d}x$,记 $W=\{p(0)=0\vert p\in V\}$,(1) 求 $W$ 的正交补;(2) 求 $f\in V$ 使得 $\langle p,f \rangle = p(0)$ 对 $\forall p\in V$ 都成立.

\noindent{\heiti\textbf{三、(12 分)}} $A\in\mathbb{C}^{7\times 7}$,且存在 $B\in\mathbb{C}^{7\times 7}$,$X\in\mathbb{C}^{7\times 1}$ 使得 $A=B^2$,$A^3 X\neq O$,$A^4 X=O$,证明 $r(A)=5$.

\noindent{\heiti\textbf{四、(12 分)}} 设 $T$ 为维数为 $n\space(n\ge 3)$ 的线性空间 $V$ 上的幂零算子,且 $r(T)=1$,求 $T$ 的 Jordan 标准形,并证明 $V$ 有无穷多个二维的 $T$-不变子空间. 

\noindent{\heiti\textbf{五、(20 分)}} 试给出下列命题的真伪. 若命题为真,请给出简要证明;若命题为假,请举出反例.

\textbf{1.} $T$ 为实算子,且只有特征值 $1$ 和 $2$,则 $T^2-3T+2I=O$.

\textbf{2.} 实算子 $T$ 有非平凡子空间,则有特征值.

\textbf{3.} $U$ 为 $V$ 的一个子空间,则可取 $U$ 的补空间的一组基 $\varepsilon_1,\cdots,\varepsilon_s$,使得 $\varepsilon_1+U,\cdots,\varepsilon_s+U$ 为商空间 $V/U$ 的一组基.

\textbf{4.} $T$ 为实内积空间上的算子,且 $T$ 在某组基下有对称矩阵,则 $T$ 是自伴算子.

\textbf{5.} 设 $s$ 为 $T$ 的奇异值,则存在单位向量 $\alpha$ 使得 $\|T\alpha\|=s$.

\noindent{\heiti\textbf{六、(12 分)}} 设 $\varepsilon_1,\cdots,\varepsilon_5$ 为 $\mathbb{R}^5$ 的一组基,$T$ 为 $\mathbb{R}^5$ 上的算子且 $T(\varepsilon_i)=\varepsilon_1+\varepsilon_2+\varepsilon_3\space(i=1,2,3)$,$T(\varepsilon_j)=\varepsilon_4+\varepsilon_5\space(j=4,5)$,求 $T$ 的极小多项式与 $T^{2025618}$.

\noindent{\heiti\textbf{七、(12 分)}} 对于算子 $T$,记 $G=T^*T$,证明以下等价:

\textbf{(1)} $T$ 是正规算子;

\textbf{(2)} 存在等距同构 $S$ 使得对任意向量 $v$,$Tv=\displaystyle\sum_{i=1}^m\displaystyle\sum_{j=1}^{d_i}\sqrt{\lambda_i}\langle v,\varepsilon_{ij}\rangle S\varepsilon_{ij}$,其中 $\{\varepsilon_{ij}\}_{j=1}^{d_i}$ 是 $E(\lambda_i,G)$ 的一组基,且 $E(\lambda_i,G)$ 是 $S$ 不变的.

\textbf{(3)} 存在等距同构 $S'$ 使得 $T^2=S'G$,且 $\operatorname{Im} T = (\operatorname{Ker} T)^\perp$.

\end{document}