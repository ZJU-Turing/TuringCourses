\documentclass{exam}

\usepackage[letterpaper,top=2cm,bottom=2cm,left=3cm,right=3cm,marginparwidth=1.75cm]{geometry}

\usepackage{amsmath}
\usepackage{amssymb}
\usepackage{zhnumber}
\usepackage{ctex}

\pointname{ 分}
\pointformat{(\thepoints)}

\title{数学分析(甲)I(H)2023秋冬期末}
\author{图灵回忆卷}

\begin{document}
\maketitle

\begin{questions}
    \question[10]叙述数列收敛的柯西收敛准则;并用该准则证明:

    \begin{center}
        数列 \(\left\{\displaystyle\sum\limits_{k=1}^n(-1)^{k+1}\dfrac{1}{k^{2024}}\right\}\) 收敛.
    \end{center}

    \question[35]计算题

    \begin{parts}
        \part 求极限 \(\lim\limits_{n\to+\infty}n\displaystyle\sum\limits_{k=1}^n\dfrac{1}{n^2+k^2}\).

        \part 求极限 \(\lim\limits_{x\to1}\dfrac{x^x-x}{1-x+\ln x}\).

        \part 求极限 \(\lim\limits_{x\to0}\dfrac{\displaystyle\int_0^{x^2}t^2e^{\sin t}\mathrm{d}t}{\ln(1+x^6)}\).
        
        \part 求 \(\displaystyle\int_{1}^{+\infty}\dfrac{\arctan x}{x^2}\mathrm{d}x\).
        
        \part 求双纽线 \(r=\sqrt{\cos(2\theta)}\) 所围平面图形的面积.
    \end{parts}

    \question[10]证明 Cantor 定理:若 \(f(x)\) 在 \([0,1]\) 上连续,则 \(f(x)\) 在 \([0,1]\) 上一致连续.

    \question[10]求函数 \(f(x)=\displaystyle\int_{-1}^{1}|x-t|e^{t^2}\mathrm{d}t\) 在 \(\mathbb{R}\) 上的最小值.

    \question[10]证明导函数介值定理:若函数 \(f(x)\) 在 \(\mathbb{R}\) 上可导,且 \(f'(0)<a<f'(1)\),则存在 \(\xi\in(0,1)\),使得 \(f'(\xi)=a\).

    \question[10]设函数 \(f(x)\) 在 \([0,1]\) 上黎曼可积,且 \(f\) 在 \(x=0\) 处右连续. 证明:函数 \(\varphi(x)=\displaystyle\int_{0}^{x}f(t)\mathrm{d}t\) \((0\leq x\leq1)\) 在 \(x=0\) 处的右导数等于 \(f(0)\).

    \question[10]设 \(f(x)\) 在 \([0,1]\) 上有连续的导函数,证明:

    \begin{center}
        \(\lim\limits_{n\to+\infty}\displaystyle\sum\limits_{k=1}^n\left(f(\dfrac{k}{n})-f(\dfrac{2k-1}{2n})\right)=\dfrac{f(1)-f(0)}{2}\).
    \end{center}

    \question[5]设函数 \(f(x)\) 在 \(\mathbb{R}\) 上有二阶连续的导函数,且存在常数 \(C>0\),使得

    \begin{center}
        \(\sup\limits_{x\in\mathbb{R}}\Big(|x|^2|f(x)|+|f''(x)|\Big)\leq C\).
    \end{center}
    
    证明:存在常数 \(M>0\),使得 \(\sup\limits_{x\in\mathbb{R}}(|xf'(x)|)\leq M\).

\end{questions}

\end{document}