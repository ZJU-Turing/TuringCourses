\documentclass[UTF8,14pt,normal]{ctexart}
\linespread{1.5}
\usepackage{amsmath}
\usepackage{geometry}
\geometry{a4paper,scale=0.66,top=0.2in,bottom=1.5in,left=1in,right=1in}

\title{\bf 数学分析(甲)I(H)2024秋冬期末}
\author{图灵回忆卷}
\date{2025年2月8日}

\begin{document}
    \maketitle 
    
    \noindent{\heiti\textbf{一、(40分)}} 计算:\vspace{1em}

    \textbf{1.} $\lim\limits_{n\to\infty}\dfrac{1}{n}\displaystyle\sum\limits_{k=1}^{n}\ln\left(1+\dfrac{k}{n}\right)$.\vspace{1em}

    \textbf{2.} $\lim\limits_{x\to 0}\dfrac{\displaystyle\int_0^{x^2}(\sin{\sqrt{t}})^2 \mathrm{d}t}{x^{4}}$.\vspace{1em}

    \textbf{3.} $f(x)=\displaystyle\int_0^x (1+t)\arctan t \,\mathrm{d}t$,求 $f(x)$ 的极值.\vspace{1em}

    \textbf{4.} 求由如下方程:
            \(
            \begin{cases}
                e^{x}=\sin t+2t+1\\
                t\sin y-y+\dfrac{\pi}{2}=0
            \end{cases}
            \)
            确定的 $y,x$ 所对应的 $\dfrac{\mathrm{d}y}{\mathrm{d}x}\vert_{x = 0}$.\vspace{1em}

    \textbf{5.} $\displaystyle\int_0^{+\infty}\dfrac{xe^{x}}{(1+e^x)^2} \,\mathrm{d}x$.\vspace{1em}

    \noindent{\heiti\textbf{二、(8分)}} 叙述确界原理,并用确界原理证明:定义在 $(0,1)$ 上的单增函数的间断点只能是跳跃间断点.\vspace{0.5em}

    \noindent{\heiti\textbf{三、(10分)}} 数列 $\{x_n\}$ 满足:$0<x_1<\dfrac{1}{2},x_{n+1}=\dfrac{1}{4(1-x_n)},\forall n \in \mathbf{N_+}$,证明:$\{x_n\}$ 收敛,且 $\lim\limits_{n\to\infty}x_n=\dfrac{1}{2}$.\vspace{0.5em}

    \noindent{\heiti\textbf{四、(10分)}} $f\in C[0,2025]$,且 $f(0)=f(2025)=0,f'_+(0)>0,f'_-(2025)>0$,
    证明:至少存在一个 $\xi$ 使得 $\xi\in(0,2025)$ 且 $f(\xi)=0$.\vspace{0.5em}
    
    \noindent{\heiti\textbf{五、(10分)}} 叙述一致连续定义,并证明:$f(x)=\sqrt{x}\ln x$ 在 $(0,+\infty)$ 上一致连续.\vspace{0.5em}
    
    \noindent{\heiti\textbf{六、(10分)}} $f(x)$ 在 $\mathbf{R}$ 上有连续二阶导数,且 $f''(x)<0$,证明:$\displaystyle\int_0^1 f(x^2) \,\mathrm{d}x \leq f(\dfrac{1}{3})$.\vspace{0.5em}
    
    \noindent{\heiti\textbf{七、(12分)}} $f\in C[0,1]$,证明:\vspace{1em}

    \textbf{1.} 存在 $\xi\in[0,\dfrac{1}{2}]$,使得 $\dfrac{1}{2}\displaystyle\int_0^1 f(x) \,\mathrm{d}x=\displaystyle\int_0^{\xi} f(x) \,\mathrm{d}x+\displaystyle\int_{1-\xi}^1 f(x) \,\mathrm{d}x$.\vspace{1em}

    \textbf{2.} 将 $\xi\in[0,\dfrac{1}{2}]$ 改为 $\xi\in (0,\dfrac{1}{2})$,结论是否仍成立?证明或给出反例.
    
\end{document}