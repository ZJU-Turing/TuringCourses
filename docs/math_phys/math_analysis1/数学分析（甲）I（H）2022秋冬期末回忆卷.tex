\documentclass[UTF8,14pt,normal]{ctexart}
\linespread{1.5}
\usepackage{amsmath}
\usepackage{geometry}
\geometry{a4paper,scale=0.66,top=0.2in,bottom=1.5in,left=1in,right=1in}

\title{\bf 数学分析(甲)I(H)2022秋冬期末}
\author{图灵回忆卷}
\date{2023年2月18日}

\begin{document}
    \maketitle 
    
    \noindent{\heiti\textbf{一、(30分)}}计算:\vspace{1em}

    \textbf{1.}$\displaystyle\lim_{x \to 1}(\dfrac{1}{x -1 } - \dfrac{1}{\ln x})$.\vspace{1em}

    \textbf{2.}$\displaystyle\lim_{n \to +\infty}\dfrac{1}{n}\sum_{k=1}^n \dfrac{\cos(\dfrac kn)}{1 + \sin(\dfrac kn)}$.\vspace{1em}

    % \textbf{3.}$f(x)=\begin{cases}\dfrac{\sin x}{x}, &x\neq 0\\1, &x=0\\\end{cases}$,求$f'(0)$和$f''(0)$.\vspace{1em}
    \textbf{3.}不定积分 $\displaystyle\int \dfrac{\ln(x + 1)}{(x + 2) ^ 2}\ \mathrm dx$.\vspace{1em}

    \textbf{4.}$y=\displaystyle\int_{-\sqrt3}^x\sqrt{3 - t^2}\ \mathrm dt$在$x \in [-\sqrt3,\ \sqrt3]$上的弧长.\vspace{1em}

    \textbf{5.}反常积分$\displaystyle\int_{0}^{+\infty}e^{-x}\cos x\ \mathrm dx$.\vspace{1em}

    \noindent{\heiti\textbf{二、(10分)}}叙述确界原理,并使用确界原理证明:有界函数$f(x)$在$(0,\ 1)$上单调递增,则极限$\displaystyle\lim_{x \to 1-} f(x)$存在.\vspace{0.5em}

    \noindent{\heiti\textbf{三、(12分)}}已知 $g(x)$ 有二阶连续导数,$g(0) = 1$,$g'(0)=0$,且 $f(x) = \begin{cases}\dfrac{g(x) - \cos x}{x}&x \neq 0\\a&x=0\\\end{cases}$
    
    \textbf{1.}若$f(x)$在$x=0$处连续,求$a$;

    \textbf{2.}已知$f(x)$在$x=0$处连续,讨论$f'(x)$在$x=0$处的连续性.\vspace{0.5em}

    \noindent{\heiti\textbf{四、(10分)}}叙述函数$f(x)$在区间$I$上一致连续的定义,并证明$f(x)=x^{\frac{1}{2023}}$在$[0,\ +\infty)$上一致连续.\vspace{0.5em}
    
    \noindent{\heiti\textbf{五、(10分)}}已知连续的非常值函数$f(x)$满足$\displaystyle\lim_{x\to+\infty}f(x)=f(0)$,证明$f(x)$在$[0,\ +\infty)$上有最大值或最小值.\vspace{0.5em}
    
    \noindent{\heiti\textbf{六、(10分)}}叙述闭区间套定理,并使用闭区间套定理证明闭区间上连续函数的零点存在性定理:$f(x)$在$[a,\ b]$连续,$f(a)f(b)<0$,则$\exists c \in (a,\ b)$使得$f(c)=0$.\vspace{0.5em}
    
    \noindent{\heiti\textbf{七、(10分)}}已知$f(x)$在$(-1,\ 2)$上有二阶导数,且$f'(\frac12)=0$.证明:
    $$\exists \xi \in (0,\ 1)\text{ 使得}\ |f''(\xi)| \ge 4|f(1) - f(0)|\text{.}$$
    
    \noindent{\heiti\textbf{八、(8分)}}已知$f(x)$在$[0,\ 1]$上有二阶连续导数,证明:
    $$ \displaystyle\int_0^1 x^nf(x)\ \mathrm dx = \dfrac{f(1)}n-\dfrac{f(1) + f'(1)}{n^2}+o(\dfrac1{n^2})\quad(n\to+\infty)$$

    
\end{document}
