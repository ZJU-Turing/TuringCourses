\documentclass{jhwhw}
\usepackage{amsmath}
\usepackage{amssymb}
\usepackage{graphicx}
\usepackage{physics}
\usepackage{ctex}

\title{23-24 秋冬数分期末参考答案}
\author{silvermilight}
\date{\today}

\begin{document}

    答案仅供参考,不保证完全正确,也不一定是最优解,欢迎指出错误或提出更好的解法. ---rzm

	\problem{(10 points)}
        叙述数列收敛的柯西收敛准则;并用该准则证明:
        \[
            \text{数列} \left\{ \sum_{k=1}^n \frac{(-1)^{(k+1)}}{k^{2024}} \right\}
            \text{收敛.}
        \]
    \solution
        Cauchy 收敛准则:
        \(
            \forall\varepsilon>0,\exists N>0,\forall n>N,m>N,
            \text{都有} |a_n-a_m|<\varepsilon,
            \text{则数列} \{a_n\} \text{收敛.}
        \)

        设数列 $a_n=\sum\limits_{k=1}^n \dfrac{(-1)^{(k+1)}}{k^{2024}}$ ,不妨设 $m<n$ ,则
        \begin{align*}
            |a_n-a_m| &= \left| \sum_{k=m+1}^n \frac{(-1)^{(k+1)}}{k^{2024}} \right| \\
                      &\leq \sum_{k=m+1}^n \frac{1}{k^{2024}}
                      < \sum_{k=m+1}^n \frac{1}{k^2} \\
                      &< \sum_{k=m+1}^n \left( \frac{1}{k-1}-\frac{1}{k} \right) \\
                      &= \frac{1}{m}-\frac{1}{n}
                      < \frac{1}{m}
        \end{align*}

        因此,只需令 $\dfrac{1}{m}<\varepsilon$ ,故取 $N=\left\lfloor \dfrac{1}{\varepsilon} \right\rfloor$ ,
        则 $\forall n>N,m>N,\text{都有}|a_n-a_m|<\varepsilon$,据 Cauchy 收敛定理知原命题成立.
    
    \problem{(35 points)}
        \begin{enumerate}
            \item 求极限 $\lim\limits_{n \to +\infty} n\sum\limits_{k=1}^n \dfrac{1}{n^2+k^2} $.
            \solution
            设 $f(x)=\dfrac{1}{1+x^2}$ ,则 $f(x) \in C[0,1]$ ,故 $f(x)$ 可积且有原函数 $\arctan{x}$

            由于 $\dfrac{n}{n^2+k^2}=\dfrac{1}{n}\cdot \dfrac{1}{1+\left( \frac{k}{n} \right)^2}
                                    =f\left(\frac{k}{n}\right)$

            \[
                \text{原式} = \int_{0}^{1} f(x) \,\text{d}x = \arctan{x} \,|_0^1 = \dfrac{\pi}{4}
            \]
            
            \item 求极限 $\lim\limits_{x \to 1} \dfrac{x^x-x}{1-x+\ln{x}}$.
            \solution
            令 $t=x-1$ ,则 $t \to 0$ ,用两次 L' Hospital 法则

            \begin{align*}
                \text{原式} &= \lim_{t \to 0} \frac{(t+1)^{t+1}-t-1}{-t+ln(1+t)} \\
                            &= \lim_{t \to 0} \frac{(1+\ln (t+1))(t+1)^{t+1}-1}{-1+\frac{1}{t+1}} \\
                            &= \lim_{t \to 0} \frac{((1+\ln (t+1))^2+\frac{1}{t+1})(t+1)^{t+1}}{-\frac{1}{(t+1)^2}} \\
                            &= -2
            \end{align*}

            \item 求极限 $\lim\limits_{x \to 0} \dfrac{\displaystyle\int_0^{x^2} t^2 e^{\sin t} \,\text{d}x}{\ln (1+x^6)} $.
            \solution
            设 $f(x)=\int_0^{x^2} t^2 e^{\sin t} \text{d}x$ ,由于被积函数在 $\mathbb{R}$ 上连续,知 $f(x) \in D( \mathbb{R} )$ ,且 $f'(x)=2x\cdot x^4 e^{\sin x^2}$
            
            当 $x \to 0$ 时,分子和分母都趋向0,可以使用 L' Hospital 法则

            \[
                \text{原式} = \lim_{x \to 0} \frac{2x^5 e^{\sin x^2}}{\frac{6x^5}{1+x^6}}
                            = \lim_{x \to 0} \frac{(1+x^6) e^{\sin x^2}}{3}
                            = \frac{1}{3}
            \]

            \item 求 $\displaystyle\int_1^{+\infty} \dfrac{\arctan x}{x^2} \,\text{d}x$.
            \solution
            \begin{align*}
                \text{原式} &= - \int_1^{+\infty} \arctan x \,\text{d}\left(\frac{1}{x}\right) \\
                            &= - \left( \frac{\arctan x}{x} \right)\bigg|_1^{+\infty} + \int_1^{+\infty} \frac{1}{x(1+x^2)} \,\text{d}x \\
                            &= \frac{\pi}{4} + \frac{1}{2} \int_1^{+\infty} \frac{1}{x^2 (1+x^2)} \,\text{d}x^2 \\
                            &= \frac{\pi}{4} + \frac{1}{2} \int_1^{+\infty} \left(\frac{1}{x^2}-\frac{1}{1+x^2} \right) \,\text{d}x^2 \\
                            &= \frac{\pi}{4} + \frac{1}{2} \left(\ln \frac{x^2}{1+x^2} \right)\bigg|_1^{+\infty} \\
                            &= \frac{\pi}{4} + \frac{1}{2} \ln 2
            \end{align*}

            \newpage

            \item 求双扭线 $r=\sqrt{\cos (2\theta)}$ 所围平面图形的面积.
            \solution
            \begin{figure}[htbp]
                \centering
                \includegraphics[scale=0.2]{function1.png}
                \caption{双扭线图像}
                \label{fig:function1}
            \end{figure}
            图 \ref{fig:function1} 表明双扭线关于x轴与y轴对称,故我们可以只计算位于第一象限的部分面积

            \[
                \text{原式} = 4 \int_{0}^{\frac{\pi}{4}} \frac{1}{2} \cos(2\theta) \,\text{d}\theta
                            = \left(\sin(2\theta)\right) \big|_0^{\frac{\pi}{4}}
                            = 1
            \]

        \end{enumerate}
    
    
    \problem{(10 points)}
        证明 Cantor 定理:若 $f(x)$ 在 $[0,1]$ 上连续,则 $f(x)$ 在 $[0,1]$ 上一致连续.
    \solution

    设 $E = \{ t\in (a,b] \,|\, f(x) \text{在} [a,t] \text{上一致连续} \} $.
    
    由 $f(x)\in C[0,1]$ ,据 Cauchy 收敛定理知 $\forall \varepsilon>0, \exists \delta>0, \forall x_1,x_2 \in [0,\delta), |f(x_1)-f(x_2)|<\varepsilon$,
    故 $f(x)$ 在 $[0,\delta)$ 上一致连续,所以 $\frac{\delta}{2} \in E$.
    
    $E\neq\emptyset$ 且有上界 $1$,由确界原理知 $E$ 有上确界,记 $\sup{E}=\alpha$.

    下证 $\alpha=1$:
    
    假设 $\alpha \in (0,1)$,则 $\forall \varepsilon>0$,

    由 $f(x)\in C[0,1]$ ,据 Cauchy 收敛定理,取 $\varepsilon_0=\varepsilon, \text{则} \exists \delta_0>0, \forall x_1,x_2 \in (\alpha-\delta_0,\alpha+\delta_0), |f(x_1)-f(x_2)|<\varepsilon_0$.

    由 $\alpha$ 为 $E$ 的上确界,$\exists \beta>\alpha-\frac{\delta}{2}, \beta\in E$ ,即 $f(x)$ 在 $[0,\beta]$ 上一致连续.
    取 $\varepsilon_1=\epsilon, \text{则} \exists \delta_1>0, \forall x_1,x_2 \in [0,\beta] \text{且} |x_1-x_2|<\delta_1, |f(x_1)-f(x_2)|<\varepsilon_1$

    取 $\delta=\min\{\frac{\delta_0}{2},\delta_1\}$,则 $\forall x_1,x_2 \in [0,\beta] \text{且} |x_1-x_2|<\delta$,$x_1$ 与 $x_2$ 必定同时落在 $[0,\beta]$ 内或 $(\alpha-\delta_0,\alpha+\delta_0)$ 内.
    故 $|f(x_1)-f(x_2)<\varepsilon$. 因此 $f(x)$ 在 $[0,\alpha+\frac{\delta_0}{2}]$ 上一致连续.

    由此可知 $\alpha+\frac{\delta_0}{2} \in E$,与 $\alpha=\sup E$ 矛盾. 故 $\alpha=1$,$f(x)$ 在 $[0,1]$ 上一致连续.

    \newpage
    \problem{(10 points)}
        求函数 $f(x)=\displaystyle\int_{-1}^1 |x-t|e^{t^2} \,\text{d}t$ 在 $\mathbb{R}$ 上的最小值.
    \solution
        分以下三种情况讨论:
        \begin{enumerate}
            \item $x \geq 1$,则
            \[
                f(x) = x\int_{-1}^1 e^{t^2} \text{d}t - \int_{-1}^1 te^{t^2} \text{d}t
                                = x\int_{-1}^1 e^{t^2} \text{d}t - \frac{1}{2} \left(e^{t^2}\right)\Big|_{-1}^1
                                \geq \int_{-1}^1 e^{t^2} \text{d}t
            \]
            \item $x \leq -1$,则
            \[
                f(x) = \int_{-1}^1 te^{t^2} \text{d}t - x\int_{-1}^1 e^{t^2} \text{d}t
                                = \frac{1}{2} \left(e^{t^2}\right)\Big|_{-1}^1 - x\int_{-1}^1 e^{t^2} \text{d}t
                                \geq \int_{-1}^1 e^{t^2} \text{d}t
            \]
            \item $-1<x<1$,则
            \begin{align*}
                f(x) &= x\int_{-1}^x e^{t^2} \text{d}t - \int_{-1}^x te^{t^2} \text{d}t +
                        \int_x^1 te^{t^2} \text{d}t - x\int_x^1 e^{t^2} \text{d}t \\
                     &= x \left(\int_{-1}^x e^{t^2}\,\text{d}t - \int_x^1 e^{t^2}\,\text{d}t \right) + e - e^{x^2}
            \end{align*}
            
            求导可得 $f'(x) = \displaystyle\int_{-1}^x e^{t^2}\,\text{d}t - \displaystyle\int_x^1 e^{t^2}\,\text{d}t,\quad f''(x)=2e^{x^2}>0$,故 $f(x)$ 单调递增.

            由 $y=e^{x^2}$ 为偶函数,知 $f'(0)=0$,故 $f(x)$ 在 $[-1,0]$ 上单调递减,在 $[0,1]$ 上单调递增,在 $x=0$ 处取得最小值 $e-1$.            
        \end{enumerate}
        
        由于 $\displaystyle\int_{-1}^1 e^{t^2} \,\text{d}t = 2\displaystyle\int_0^1 e^{t^2} \,\text{d}t > 2\displaystyle\int_0^1 te^{t^2} \,\text{d}t = e-1$,故 $e-1$ 为 $f(x)$ 在 $\mathbb{R}$ 上的最小值.


    \problem{(10 points)}
        证明导函数极限定理:若函数 $f(x)$ 在 $\mathbb{R}$ 上可导,且 $f'(0)<a<f'(1)$,则存在 $\xi \in (0,1)$,使得 $f'(\xi)=a$.
    \solution
        设 $g(x)=f(x)-ax$,则只需证明 $\exists \xi \in (0,1), g'(\xi)=0$.

        $g(x)=f(x)-ax \in C[0,1]$,由闭区间上连续函数的最值定理可知 $g(x)$ 在 $[0,1]$ 上有最小值.

        $g'(0)=\lim\limits_{x\to 0} \frac{g(x)-g(0)}{x}<0$,由局部保号性可知 $\exists \delta_1>0, \forall x \in (0,\delta_1), g(x)<g(0)$.

        $g'(1)=\lim\limits_{x\to 1} \frac{g(x)-g(1)}{x}>0$,由局部保号性可知 $\exists \delta_2>0, \forall x \in (1-\delta_2,1), g(x)<g(0)$.

        所以 $g(0)$ 和 $g(1)$ 都不是 $g(x)$ 在 $[0,1]$ 上的最小值,故最小值在 $(0,1)$ 上的某个点 $\xi$ 处取得,$xi$ 同时也是极小值.

        又因为 $g(x)\in D[0,1]$,由 Fermat 引理可知 $g'(\xi)=0$,得证.
    
    
    \problem{(10 points)}
        设函数 $f(x)\in R[0,1]$,且 $f$ 在 $x=0$ 处右连续. 证明:函数 $\varphi(x)=\displaystyle\int_0^x f(t) \,\text{d}t \;(0 \leq x \leq 1)$ 在 $x=0$ 处的右导数等于 $f(0)$.
    \solution
        $f(x)$ 在 $x=0$ 处右连续,故 $\forall \varepsilon>0, \exists \delta>0, \forall x \in (0,\delta), |f(x)-f(0)|<\varepsilon$,即 $f(0)-\varepsilon<f(x)<f(0)+\varepsilon$.

        由定积分保序性,可知 $\forall x \in (0,\delta), (f(0)-\varepsilon)x<\varphi(x)<(f(0)+\varepsilon)x$.

        所以 $\forall \varepsilon>0, \exists \delta>0, \forall x \in (0,\delta),  f(0)-\varepsilon<\dfrac{\varphi(x)}{x}<f(0)+\varepsilon$

        所以 $\varphi'_+(0) = \lim\limits_{x\to 0^+} \dfrac{\varphi(x)}{x} = f(0)$
    

    \problem{(10 points)}
        设 $f(x)$ 在 $[0,1]$ 上有连续的导函数,证明:
        \[
            \lim_{n \to +\infty} \sum_{k=1}^n \left[ f\left(\frac{k}{n}\right) - f\left(\frac{2k-1}{2n}\right) \right] = \frac{f(1)-f(0)}{2}
        \].
    \solution
        由条件知 $f(x) \in C[0,1] \cap D(0,1)$,据 Lagrange 中值定理,$\sum\limits_{k=1}^n \left[ f\left(\frac{k}{n}\right) - f\left(\frac{2k-1}{2n}\right) \right] = \frac{1}{2n} \sum\limits_{k=1}^n f'(\xi_k) $,
        且有 $\xi_k \in \left( \frac{2k-1}{2n},\frac{k}{n} \right) \subset \left( \frac{k-1}{n}, \frac{k}{n} \right)$

        取分割 $\Delta \colon x_0=0,x_1=\frac{1}{n},\cdots,x_k=\frac{k}{n},\cdots,x_n=1$ 与介点组 $\{\xi_k\}$,由 $f'(x) \in C[0,1]$,知 $f'(x)$ 黎曼可积且有原函数 $f(x)$.

        故 \[
            \lim_{n \to +\infty} \sum_{k=1}^n \left[ f\left(\frac{k}{n}\right) - f\left(\frac{2k-1}{2n}\right) \right]
            = \frac{1}{2} \lim_{\| \Delta \| \to 0} \sum_{k=1}^{n} f'\left(\xi_k\right) \frac{1}{n}
            = \frac{1}{2} \int_{0}^{1} f'(x) \,\text{d}x
            = \frac{f(1)-f(0)}{2}
        \]
    
    
    \problem{(5 points)}
        设函数 $f(x)$ 在 $\mathbb{R}$ 上有二阶连续的导函数,且存在常数 $C>0$,使得
        \[
            \sup_{x \in \mathbb{R}} \left( |x|^2 |f(x)| + |f''(x)| \right) \leq C
        \].

        证明:存在常数 $M>0$,使得 $\sup\limits_{x \in \mathbb{R}} \left(|xf'(x)|\right) \leq M$.
        
    \solution
        由上确界定义可知 $\forall x \in \mathbb{R}, |x|^2 |f(x)|\leq C, \; |f''(x)|\leq C$.
        \begin{enumerate}
            \item $|x|\leq 1$

                $f(x)$ 有二阶连续导数,故 $f'(x) \in C[-1,1]$,进而 $xf'(x) \in C[-1,1]$,
                由有界性定理可知 $\exists M_1>0, \forall x \in [-1,1], |xf'(x)|\leq M_1$.

            \item $|x|>1$
            
                $f(x)$ 有二阶连续导数,$\forall |x_0|>1$,\begin{equation}
                    f(x) = f(x_0) + f'(x_0)(x-x_0) + \frac{f''(\xi)}{2}(x-x_0)^2
                \end{equation}
                
                代入 $x=x_0+\dfrac{1}{x_0}$,有 \begin{equation}
                    f(x_0+\frac{1}{x_0}) = f(x_0) + \frac{f'(x_0)}{x_0} + \frac{f''(\xi)}{2x_0^2}
                \end{equation}

                两边同乘 $x_0^2$,移项得 \begin{equation}
                    x_0 f'(x_0) = x_0^2 f(x_0+\frac{1}{x_0}) - x_0^2 f(x_0) - \frac{f''(\xi)}{2}  \label{eq:result1}
                \end{equation}

                等式右边第二项与第三项都是有界量,故只需证明第一项也为有界量,考虑利用有界量 $(x_0+\frac{1}{x_0})^2 f(x_0+\frac{1}{x_0})$ 进行估计: \begin{align}
                    \left| x_0^2 f(x_0+\frac{1}{x_0}) \right| &= \left| (x_0+\frac{1}{x_0})^2 f(x_0+\frac{1}{x_0}) - 2f(x_0+\frac{1}{x_0}) - \frac{f(x_0+\frac{1}{x_0})}{x_0^2} \right| \\
                                                              &\leq \left| (x_0+\frac{1}{x_0})^2 f(x_0+\frac{1}{x_0}) \right| + \left| 2f(x_0+\frac{1}{x_0}) \right| + \left| \frac{f(x_0+\frac{1}{x_0})}{x_0^2} \right| \\
                                                              &< \left| (x_0+\frac{1}{x_0})^2 f(x_0+\frac{1}{x_0}) \right| + 3 \left| x_0^2 f(x_0+\frac{1}{x_0}) \right| \\
                                                              &\leq 4C \label{eq:result2}
                \end{align}

            将\eqref{eq:result1}式取绝对值,再将\eqref{eq:result2}式代入,得 \begin{align*}
                \left| x_0 f'(x_0) \right| &\leq \left| x_0^2 f(x_0+\frac{1}{x_0}) \right| + \left| x_0^2 f(x_0) \right| + \frac{\left|f''(\xi)\right|}{2} \\
                                           &\leq 4C + C + \frac{C}{2} \\
                                           &= \frac{11C}{2}
            \end{align*}
        \end{enumerate}

        综上,取 $M=\max\left\{M_1,\frac{11C}{2}\right\}$,则 $\forall x\in \mathbb{R}, |xf'(x)|\leq M$,
        故 $M$ 为 $|xf'(x)|$ 的一个上界,必然不小于其上确界,因此 $\sup\limits_{x \in \mathbb{R}} \left(|xf'(x)|\right) \leq M$.

\end{document}
