\documentclass{article}
\usepackage{ctex}
\usepackage{amsmath}
\usepackage{geometry}
\usepackage{amsfonts,amssymb}
\geometry{a4paper,left=2cm,right=2cm,top=4cm}
\title{Discrete Mathematics Quiz 1\\\small{2024 - 2025 春夏学期}}
\author{5dbwat4}
% By 5dbwat4<me@5dbwat4.top>
\date{\today}
\setlength{\parskip}{1em}
\setlength{\baselineskip}{20pt}
\setlength{\parindent}{0em}
\begin{document}

\maketitle


\begin{enumerate}
\def\labelenumi{\arabic{enumi}.}
\item
  \emph{(35\%)} Determine whether the following statements are true or
  false. (\textbf{5 points} for a correct answer, \textbf{0 points} for
  a blank answer, \textbf{-2 points for an incorrect answer})

\begin{enumerate}
\def\labelenumi{\alph{enumi})}
\item
  If \(x\) is not occurring in \(A\), then
  \(\exists x(P(x)\to A)\equiv \forall x P(x) \to A\).
\item
  If \(A\), \(B\), and \(C\) are sets, then
  \(A-(B\cap C)=(A - B)\cup(A - C)\).
\item
  If \(n\) is integer, then
  \(n=\left\lceil\frac{n}{2}\right\rceil+\left\lfloor\frac{n}{2}\right\rfloor\).
\item
  Suppose \(P(x,y)\) is a predicate and the universe for the variables
  \(x\) and \(y\) is \(\{1,2,3,4\}\). Suppose \(P(1,3)\), \(P(2,1)\), \(P(2,4)\),
  \(P(3,2)\), \(P(3,4)\), \(P(4,1)\), \(P(4,4)\) are true, and
  \(P(x,y)\) is false otherwise. Then the statement
  \(\forall x\exists y((x\leq y)\land P(x,y))\) is true.
\item
  \(n^{0.01}\) is \(O(\log_{1.01}{n})^{99999}\).
\item
  The set of positive real numbers less than 1 with decimal
  representations consisting only of 0s and 1s is countable.
\item
  \(2025^{2026}\equiv1\pmod{2027}\).
\end{enumerate}

\item
  \emph{(12\%)} Write a proposition equivalent to \(p\oplus q\),

\begin{enumerate}
\def\labelenumi{\alph{enumi})}
\item
  using only \(p\), \(q\), \(\neg\), and the connective \(\land\).
\item
  using only \(p\), \(q\), and the connective \(|\) (``\(|\)''
  represents NAND 与非).
\end{enumerate}

\item
  \emph{(9\%)} Find the full conjunctive normal form of
  \((p\oplus q)\lor r\).
\item
  \emph{(8\%)} Build all the functions from \(A = \{1,2\}\) to
  \(B=\{a,b\}\) and point out which is bijection, and which is
  surjection.
\item
  \emph{(9\%)} If all the positive integers that are relatively prime
  with 77 are arranged into a strictly increasing sequence, find the
  600\emph{th} term of this sequence.
\item
  \emph{(9\%)} Use the construction in the proof of the Chinese
  remainder theorem to find all solutions to the system of congruences
  \(x\equiv1\pmod{3}\), \(x\equiv2\pmod{5}\), and \(x\equiv3\pmod{8}\).
\item
  \emph{(9\%)} Prove that the distributive law
  \(A_1\cup(A_2\cap\cdots\cap A_n)=(A_1\cup A_2)\cap\cdots\cap(A_1\cup A_n)\)
  is true for all \(n > 2\).
\item
  \emph{(9\%)} Prove that every positive integer (\(n>2\)) can be
  expressed as the sum of different Fibonacci numbers.
\end{enumerate}


\end{document}  